\chapter{Considera��es Finais}
\label{cap:estudoCaso}

\section{Conclus�o}
\label{sec:conclusao}

Com este aplicativo desenvolvido a empresa ICF ira atingir maior n�meros de
de recrutamentos de participantes de pesquisa para seus estudos com baixo custo
de divulga��o de informativos de estudos.
E devido hoje em dia os \textit{smartphones} est�o cada vez mas presentes na vida das pessoas, 
com uma aplicativo rodando no telefone fica muito mais facil contato com estas
pessoas. Facilitando a visualiza��o da notifica��es no seu proprio celular.


\subsection{Contribui��o}
\label{subsec:contribuicao}
Neste trabalho foi poss�vel desenvolver um  aplicatico para recrutamento de
participantes de pesquisas.
\begin{itemize}
\item
Foi estudado o referencial te�rico para o desenvolvimento de um aplicativo
adroid;
\item
Recolhido os requisitos funcionais e n�o funcionais;
\item
Foi realizado o projeto de Banco de Dados;
\item
Implementado um aplicativo denominado R2P;
\item
Foi criado o caso de uso, diagrama de classes e diagrama de sequ�ncia;
\item
Os resultados foram analisados de acordo com o que foi pedido.
\end{itemize}

\section{Trabalhos Futuros}
\label{sec:trabalhosFuturos}

Antes, sem um aplicativo para notificar os participantes de pesquisa para
participarem de um estudo era feito atrav�z de aquisi��es de pacotes de sms para
envio nos celulares dos mesmos. O que gerava custos altos a empresa.
Agora com o aplicativo a empresa envia notifica��es as pessoas a custo zero.

Para trabalhos futuros ser� criado uma tela para intera��es entre a empresa e os
participantes de pesquisa que ele ir�o enviar d�vidas com o apicativo R2P e a
empresa ir� responder atrav�z de um sistema coorporativo.
Ser� feito melhorias de visual das telas do aplicativo. Implenta��o do
\textit{Push Notifications}, ou seja toda vez que a empresa ICF cadastrar uma
notifica��o nova o participante de pesquisa ir� receber bip em seu celular
avisando de novo estudo dispon�vel para sua participa��o.
O estudo � cadastro no sistema corporativo da empresa ICF cadas estudo tem uma
identifica��o com PBIO010/11 que n�o pode ser divultado para os partivipantes de
pesquisa devido uma determi��o da ANVISA, por isso n�o ser� divulgado nas
mensagens enviadas.



 